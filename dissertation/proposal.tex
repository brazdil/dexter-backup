%%%%%%%%%%%%%%% Updated by MR March 2007 %%%%%%%%%%%%%%%%
\documentclass[12pt]{article}
\usepackage{a4wide}
\usepackage[UKenglish]{babel}
\usepackage[UKenglish]{isodate}
\usepackage[style=footnote-dw]{biblatex}
\bibliography{memoire-bb}

\newcommand{\al}{$<$}
\newcommand{\ar}{$>$}

\parindent 0pt
\parskip 6pt

\begin{document}

\thispagestyle{empty}

\rightline{\large\emph{David Brazdil}}
\medskip
\rightline{\large\emph{Trinity Hall}}
\medskip
\rightline{\large\emph{db538}}

\vfil

\centerline{\large Computer Science Tripos Part II Project Proposal}
\vspace{0.4in}
\centerline{\Large\bf Taint-Based Flow Tracing by}
\vspace{0.1in}
\centerline{\Large\bf Bytecode Instrumentation on Android}
\vspace{0.3in}
\cleanlookdateon
\centerline{\large \emph \today}

\vfil

{\bf Project Originator:} \emph{Dr A. R. Beresford}

%\vspace{0.1in}
%
%{\bf Resources Required:} See attached Project Resource Form

\vspace{0.5in}

{\bf Project Supervisor:} \emph{Dr A. R. Beresford}

\vspace{0.2in}

{\bf Signature:}

\vspace{0.5in}

{\bf Director of Studies:}  \emph{Dr S. W. Moore}

\vspace{0.2in}

{\bf Signature:}

\vspace{0.5in}

{\bf Overseers:} \emph{Prof J. M. Bacon}\ and \emph{Prof R. J. Anderson}

\vspace{0.2in}

{\bf Signatures:} 

\vfil
\eject

\section*{Introduction and Description of the Work}

Android is a popular, open-source Linux-based platform for touchscreen
mobile devices, such as smartphones and tablets. Ever since its unveiling 
in 2007, it has been increasing its market share, running on 68.1\% of 
smartphones sold in the second quarter of 2012 according to a report by 
the International Data Corporation. 
\cite{www.idc.com/getdoc.jsp?containerId=prUS23638712}

Unfortunately, with its growing popularity, the platform became a frequent
target of increasingly sophisticated malware, masquerading as legitimate
software, while leaking sensitive data about the users. More complex
malicious applications even try to hide their activity by exploiting 
security holes of the underlying operating-system level to bypass the 
protection mechanisms of the platform.

Researchers have been coming up with different approaches to solving
this problem, mostly focusing on porting well-established methods from
the desktop environment to resource-limited devices running Android and
other competing mobile operating systems. These methods range from 
lightweight static analysis of the executable code, all the way to 
enforcing sandboxing by virtualization. 

One such approach is shown in TaintDroid \cite{www.appanalysis.org}, 
project developed by a group of researchers from The Pennsylvania State
University, Duke University and Intel Labs. They point out that the
access control in Android is rather coarse-grained, with all-or-nothing
policy. This means that once an application gets an access to a resource,
it can do whatever it wants with it, sometimes not only without the 
user's consent, but also without his or her knowledge.

TaintDroid modifies several core parts of the Android platform, such as 
the Dalvik VM, the Android shared library, and even the file-system. 
Id adds support for runtime labelling (tainting) of sensitive data like 
the phone number, the contact list or GPS location, and tracing the flow 
of such data through the system. By checking the labels of data that are
leaving the system, e.g. via the network connection, TaintDroid can warn 
the users about the possibility of their data being misused. It is 
therefore an analytical tool providing insight into the behaviour of 
third-party applications, giving the users a better picture of what 
happens with the data they entrust to their applications.

Following the work conducted on TaintDroid, the outcome of this project 
will be a desktop application with similar goals, but achieving them in 
a different manner. TaintDroid integrates into the lower levels of the
operating system, altering Dalvik's memory management and instructions 
to store and propagate the taint transparently to the applications 
running on top of it. However, similar result can be accomplished by 
extracting the executable code from the application's package and 
instrumenting it to carry out the information-flow analysis itself, 
without any modification to the platform needed, which is what this 
project will try to attain.

Even though the resulting desktop application will be intended mostly for
use by professionals, the configuration of the privacy policies should be
intuitive enough even for users without deep understanding of taint-based
analysis. Typical user will:
\begin{itemize}
\item{connect their Android device to the computer}
\item{choose an installed application which should be instrumented}
\item{select sources of information and output channels to be monitored}
\item{wait for the application to get instrumented, repackaged and sent
      back to the device}
\item{run the application and wait for notifications about privacy policy
      violations}
\end{itemize}

Limitations and advantages of both solutions will be thoroughly 
compared in the evaluation section of the dissertation. Obviously, 
the low-level method makes it possible to trace taint throughout 
the system by patching the IPC kernel module, or to store taint within 
attributes of files. This can't be done by per-application bytecode 
instrumentation, but the fact that each application is processed before 
it is loaded back into the device and executed leaves room for static 
analysis of the code, possibly even alteration of its behaviour. 

\section*{Resources Required}

Development will require a computer with the Android SDK 
\cite{developer.android.com} installed. Most of the work will be done
on my personal computer, but testing on a large sample of applications
will need to be executed on a PWF machine. 

Large collection of infected applications have been obtained from the
Android Malware Genome Project \cite{www.malgenomeproject.org}. 
Sufficient disc allocation of 3GB will be needed on the PWF to store
a snapshot of the repository. 

\emph{PHONE}

\section*{Starting Point}

\al\emph{This is the place to declare any prior knowledge relevant to
  the project.  For example any relevant courses taken prior to the
  start of the Diploma year.}\ar


\section*{Substance and Structure of the Project}

Dynamic data-flow analysis will form the core of the project.

, but based
on other work done in this area, there are several ways of extending it.

\begin{description}
\item{Implicit-flow analysis} \\
TaintDroid focuses strictly on tracing the explicit flows of information,
by propagating the taint when data-handling CPU instructions are used. 
But information can be leaked via implicit flows (branching instructions)
as well. Consider the following example:
\begin{verbatim}
int sensitiveData = getSensitiveData(); // tainted variable
int leakedData = 0; // untainted variable

while (sensitiveData--)
    leakedData++;

output(leakedData); // leakedData still not tainted
\end{verbatim}

In this case, \verb|leakedData| has not been tainted, because there is 
no explicit information flow from \verb|sensitiveData|. Instead, 
\verb|sensitiveData| effects the control flow of the program, leaking
the information into \verb|leakedData| implicitly. 

Leakage via implicit flow is best identified by static analysis of the 
source code of the program, often used to analyse scripting languages. 
Since the source code is not available for third-party applications on 
Android, it needs to be analysed dynamically on the instruction level.
Downside of this solution is that the taint-propagation rules must be
rather conservative, leading to false positives. This is the reason why
TaintDroid developers decided not to include it into their system-wide
solution. Since this project will instrument applications separately, 
the user will be given the choice of instrumentation including or 
excluding implicit-flow analysis.

\item{Causality analysis} \\
Lorem ipsum dolor sit amet, consectetur adipiscing elit. Duis fringilla facilisis rutrum. Ut fringilla, erat sit amet tincidunt scelerisque, dolor magna semper orci, eu blandit mi tellus quis nisi. Maecenas massa purus, elementum vel scelerisque luctus, convallis eget est. Cras vestibulum turpis ac lorem euismod vel rhoncus nulla imperdiet. Praesent molestie dolor eget mauris mollis sit amet adipiscing ligula consequat. Ut nibh risus, pulvinar sed viverra ac, consequat ut ipsum. Sed lacinia, libero id egestas consequat, erat purus hendrerit eros, nec adipiscing mi magna ut felis.

\item{Mocking} \\
Lorem ipsum dolor sit amet, consectetur adipiscing elit. Duis fringilla facilisis rutrum. Ut fringilla, erat sit amet tincidunt scelerisque, dolor magna semper orci, eu blandit mi tellus quis nisi. Maecenas massa purus, elementum vel scelerisque luctus, convallis eget est. Cras vestibulum turpis ac lorem euismod vel rhoncus nulla imperdiet. Praesent molestie dolor eget mauris mollis sit amet adipiscing ligula consequat. Ut nibh risus, pulvinar sed viverra ac, consequat ut ipsum. Sed lacinia, libero id egestas consequat, erat purus hendrerit eros, nec adipiscing mi magna ut felis.
 
\item{Performance optimization} \\
Lorem ipsum dolor sit amet, consectetur adipiscing elit. Duis fringilla facilisis rutrum. Ut fringilla, erat sit amet tincidunt scelerisque, dolor magna semper orci, eu blandit mi tellus quis nisi. Maecenas massa purus, elementum vel scelerisque luctus, convallis eget est. Cras vestibulum turpis ac lorem euismod vel rhoncus nulla imperdiet. Praesent molestie dolor eget mauris mollis sit amet adipiscing ligula consequat. Ut nibh risus, pulvinar sed viverra ac, consequat ut ipsum. Sed lacinia, libero id egestas consequat, erat purus hendrerit eros, nec adipiscing mi magna ut felis.

\item{Reflection} \\
One major limitation of analysis by bytecode instrumentation is that it 
cannot easily deal with reflection. 

\end{description}





\section*{Success Criterion}

For the project to be deemed a success the following items must be
successfully completed.

\begin{enumerate}

\item A notation in the style of ladder logic needs to be designed and
  specified in detail.

\item An internal representation of the ladder logic program must be
  designed.

\item Program/programs must be designed and implemented to convert
  this notation into a form suitable for interpretation/simulation.

\item The data structures and algorithms used by the discrete event simulator
must be designed and implemented.

\item Details of what output the system should generate and how it
  should be controlled (by the user) must be specified.

\item Demonstration test programs should be written and run to
  demonstrate the the project works.

\item Some indication of the efficiency of the implementation in terms
  of space and time should be given.

\item The dissertation must be planned and written.

\end{enumerate}

\al\emph{The sketches above are to be taken as starting points for
  investigation of the literature and discussion with your Supervisor.
  A look at the 1996 Diploma Dissertation, {\rm A Ladder Logic
    Compiler and Interpreter}, by Johnson Adesanya, is likely to prove
  useful since it provides some pointers into the literature, and
  shows in detail one way of solving the problem}.\ar

\medskip \al\emph{It would also be possible to adjust the style and
  emphasis of the project either towards the best possible
  computational capability for your code, or in the direction of
  better human/machine interaction, particularly for the visualisation
  of the behaviour of given circuit designs}.\ar



\section*{Timetable and Milestones}

\al\emph{In the following scheme, weeks are numbered so that the week
  starting on the day on which Project Proposals are handed in is
  Week~1.  The year's timetable means that the deadline for submitting
  dissertations is in Week~34.}\ar

\al\emph{In the Project Proposal that you hand in, {\rm actual dates}
  should be used instead of week numbers and you should show how these
  dates relate to the periods in which lectures take place. Week~1
  starts immediately after submission of the Project Proposal.}\ar

\al\emph{The timetable and milestones given below refer to just one
  particular interpretation of this document.  Even if you select
  exactly this interpretation you will need to review the suggested
  timetable and adjust the dates to allow as precisely as you can for
  the amount of programming and other related experience that you have
  at the start of the year.  Take account of the dates you and your
  Supervisor will be working in Cambridge outside Lecture Term.  Note
  that some candidates write the Introduction and Preparation chapters
  of their dissertations quite early in the year, while others will do
  all their writing in one burst near the end}.\ar


\subsection*{Before Proposal submission}

\al\emph{This section will not appear in your Project Proposal.}\ar
 
Submission of Phase~1 Report Form. Discussion with Overseers and
Director of Studies.  Allocation of and discussion with Project
Supervisor, preliminary reading, choice of the variant on the project
and language \al\emph{Java in this example\/}\ar, writing Project
Proposal.  Discussion with Supervisor to arrange a schedule of regular
meetings for obtaining support during the course of the year.

Milestones: Phase~1 Report Form (on the Monday immediately following
the main Briefing Lecture), then a Project Proposal complete with as
realistic a timetable as possible, approval from Overseers and
confirmed availability of any special resources needed. Signatures
from Supervisor and Director of Studies.


\subsection*{Weeks 1 to 5}

\al\emph{Real work on the project starts here (as distinct from just
  work on the proposal).  A significant problem for Diploma candidates
  is that this critical period largely coincides with the Christmas
  vacation.  There is no guarantee that supervisors will be available
  outside Lecture Term, but Diploma students take much less of a
  Christmas break than undergraduates do, and so have some opportunity
  for uninterrupted reading and initial practical work at this stage.
  It is important to have completed some serious work on the project
  before the pressures of the Lent Term become all too apparent.}\ar

Study C and the particular implementation of it to be used.  Practise
writing various small programs, including key fragments of the compiler
and interpreter.

Milestones: Some working example C programs including code to deal
with symbol tables in the lexical analyser, and part implementation of
the time queue to be used in the interpreter/simulator.


\subsection*{Weeks 6 and 7}

Further literature study and discussion with Supervisor to ensure that
the chosen data structures are satisfactory.  Implementation of the
syntax analyser and debugging code to help test it.  This is likely to
be code that can be used to display the data structures in
human-readable form so that it is possible to check that they are as
expected.

Milestones: Ability to construct and display data structures that
represent simple ladder logic programs such as:

\begin{verbatim}
    |    A               B                C     |
    +---| |-------------| |--------------( )----+
\end{verbatim}


\subsection*{Weeks 8 to 10}

Implementation of the translation phase of the compiler. This will of
a decision to be made on what target code to generate. The obvious contnders
are: an interpretive code, Java or C.

Start to plan the Dissertation, thinking ahead especially to the
collection of examples and tests that will be used to demonstrate that
the project has been a success. 

Milestones: Ability to compile some very simple ladder logic programs,
and print out some human readable version of the target code produced.


\subsection*{Weeks 11 and 12}

Complete code for the interpreter/simulator, or the environment in
which to run the generated Java or C target code.

Prepare further test cases.  Review timetable for the remainder of the
project and adjust in the light of experience thus far.  Write the
Progress Report drawing attention to the code already written,
incorporating some examples, and recording any augmentations which at
this stage seem reasonably likely to be incorporated.

Milestones: Simple ladder logic programs should now compile and run
correctly, but probably with some serious inefficiencies in the code.
Progress Report submitted and entire project reviewed both personally
and with Overseers.


\subsection*{Weeks 13 to 19 (including Easter vacation)}

Rework the entire implementation to enhance the
richness of the language it can deal with. \al\emph{It is possible that
the initial implementation has severe restrictions on the number of
relays or coils that can be handled. Such restrictions can be freed at
this stage.}\ar\  Write initial chapters of the Dissertation.

\al\emph{The Easter break from lectures can provide a time to work on a
substantial challenge such as the computation of logarithms, where an
uninterrupted week can allow you to get to grips with a fairly
complicated algorithm.  This is a good time to put in some quiet work
(while your Supervisor is busy on other things) writing the
Preparation and Implementation chapters of the Dissertation.  By this
stage the form of the final implementation should be sufficiently
clear that most of that chapter can be written, even if the code is
incomplete.  Describing clearly what the code will do can often be a
way of sharpening your own understanding of how to implement it.}\ar

Milestones: Preparation chapter of Dissertation complete,
Implementation chapter at least half complete, code can perform a
variety of interesting tasks and should be in a state that in the
worst case it would satisfy the examiners with at most cosmetic
adjustment.


\subsection*{Weeks 20 to 26}

\al\emph{Since your project is, by now, in fairly good shape there is
a chance to use the immediate run-up to exams to attend to small
rationalisations and to implement things that are useful but fairly
straightforward.  It is generally not a good idea to drop all project
work over the revision season; if you do, the code will feel amazingly
unfamiliar when you return to it.  Equally, first priority has to go
to the exams, so do not schedule anything too demanding on the project
front here.  The fact that the Implementation chapter of the
Dissertation is in draft will mean that you should have a very clear
view of the work that remains, and so can schedule it rationally.}\ar

Work on the project will be kept ticking over during this period but
undoubtedly the Easter Term lectures and examination revision will
take priority.


\subsection*{Weeks 27 to 31}

\al\emph{Getting back to work after the examinations and May Week
  calls for discipline.  Setting a timetable can help stiffen your
  resolve!}\ar

Testing and evaluation.  Finish off otherwise ragged parts of the
code.  Write the Introduction chapter and draft the Evaluation and
Conclusions chapters of the Dissertation, complete the Implementation
chapter.

Milestones: Examples and test cases run and results collected,
Dissertation essentially complete, with large sections of it
proof-read by Supervisor and possibly friends and/or Director of
Studies.


\subsection*{Weeks 32 to 33}

Finish Dissertation, preparing diagrams for insertion.  Review whole
project, check the Dissertation, and spend a final few days on
whatever is in greatest need of attention.

\al\emph{In many cases, once a Dissertation is complete (but not
  before) it will become clear where the biggest weakness in the
  entire work is.  In some cases this will be that some feature of the
  code has not been completed or debugged, in other cases it will be
  that more sample output is needed to show the project's capabilities
  on larger test cases.  In yet other cases it will be that the
  Dissertation is not as neatly laid out or well written as would be
  ideal.  There is much to be said for reserving a small amount of
  time right at the end of the project (when your skills are most
  developed) to put in a short but intense burst of work to try to
  improve matters.  Doing this when the Dissertation is already
  complete is good: you have a clearly limited amount of time to work,
  and if your efforts fail you still have something to hand in!  If
  you succeed you may be able to replace that paragraph where you
  apologise for not getting feature X working into a brief note
  observing that you can indeed do X as well as all the other things
  you have talked about.}\ar


\subsection*{Week 34}

\al\emph{Aim to submit the dissertation at least a week before the
  deadline. Be ready to check whether you will be needed for a\/ {\rm
    viva voce} examination}.\ar

Milestone: Submission of Dissertation. 

\printbibliography
\end{document}
