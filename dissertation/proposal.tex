%%%%%%%%%%%%%%% Updated by MR March 2007 %%%%%%%%%%%%%%%%
\documentclass[12pt]{article}
\usepackage{a4wide}
\usepackage{units}
\usepackage[UKenglish]{babel}
\usepackage[UKenglish]{isodate}
%\usepackage[style=footnote-dw]{biblatex}
%\bibliography{memoire-bb}

\newcommand{\al}{$<$}
\newcommand{\ar}{$>$}

\parindent 0pt
\parskip 6pt

\begin{document}

\thispagestyle{empty}
\textheight = 650pt
\topmargin = 0pt
\voffset = -10pt

%\rightline{\large\emph{David Brazdil}}
%\medskip
%\rightline{\large\emph{Trinity Hall}}
%\medskip
%\rightline{\large\emph{db538}}
%
%\vspace{0.3in}
%\centerline{\large Computer Science Tripos Part II Project Proposal}
%\vspace{0.4in}
%\centerline{\Large\bf Taint-Based Flow Tracing by}
%\vspace{0.1in}
%\centerline{\Large\bf Bytecode Instrumentation on Android}
%\vspace{0.3in}
%\cleanlookdateon
%\centerline{\large \emph \today}
%\vspace{0.5in}

\section*{Introduction and Description of the Work}

Android is a popular open-source platform for touchscreen
mobile devices, such as smartphones and tablets. Ever since its unveiling 
in 2007, it has been increasing its market share, running on 68.1\% of 
smartphones sold in the second quarter of 2012 according to a report by 
the International Data Corporation\footnote
{www.idc.com/getdoc.jsp?containerId=prUS23638712}.

Unfortunately, with its growing popularity, the platform became a frequent
target of increasingly sophisticated malware, masquerading as legitimate
software, while leaking sensitive data about users. More complex
malicious applications even try to hide their activity by exploiting 
security holes of the underlying operating system to bypass the 
protection mechanisms of the platform.

Researchers have been coming up with different approaches to solving
this problem, mostly focusing on porting well-established methods from
the desktop environment to resource-limited devices running Android and
other competing mobile operating systems. These methods range from 
static analysis of the executable code, all the way to enforcing sandboxing 
by virtualization. 

One such approach is shown in TaintDroid\footnote{www.appanalysis.org}, 
a project developed by a group of researchers from The Pennsylvania State
University, Duke University and Intel Labs. They point out that the
access control in Android is rather coarse-grained, with an all-or-nothing
policy. This means that once an application is given access to a resource,
it can do whatever it wants with it, and does not need any further consent 
from the user.

TaintDroid modifies several core parts of the Android platform, such as 
the Dalvik VM, the Android shared library, and even the file-system, 
adding support for runtime labelling (tainting) of sensitive data like the 
phone number, the contact list or GPS location, and tracing the flow of such 
data through the system. By checking the labels of data that are leaving the 
system, e.g. via the network connection, TaintDroid can warn the users about 
the possibility of their data being misused. It is therefore an analytical 
tool providing insight into the behaviour of third-party applications, giving 
the users a better picture of what happens with the data they entrust to 
their applications.

Following the work conducted on TaintDroid, the outcome of this project 
will be a tool with similar goals, but achieving them in a different 
manner. TaintDroid integrates into the lower levels of the operating 
system, altering Dalvik's memory management and instructions to store and
propagate the taint transparently to the applications running on top of 
it. However, a similar result can be accomplished by extracting the 
executable code from the application's package and instrumenting it to 
carry out the information-flow analysis itself, without any modification 
to the platform needed, which is what this project will try to attain.

The low-level method makes it possible to trace the tainted data 
throughout the system by patching the IPC kernel module, or by storing the 
taint within attributes of files. This cannot be done by per-application 
bytecode instrumentation, but the fact that each application is processed 
before it is loaded back into the device and executed, leaves room for static 
analysis of the code, perhaps even modification of its behaviour. Limitations 
and advantages of both solutions will be thoroughly compared in the evaluation 
section of the dissertation.

Despite being intended mostly for use by professionals, the configuration 
of monitored privacy policies should be intuitive enough even for users 
without deep understanding of dynamic taint analysis. Typical user will:
\begin{itemize}
\item{connect their Android device to the computer}
\item{choose an installed application that should be instrumented}
\item{select data sources and output channels (sinks) to be monitored}
\item{wait for the application to get instrumented, repackaged and sent
      back to the device}
\item{run the application and wait for notifications about privacy policy
      violations}
\end{itemize}

\section*{Starting Point}

Even though numerous tools that help identify malware on Android have been 
developed, they have chosen different strategies of doing so. This project
will build on the strategy implemented in TaintDroid, but will also try to 
combine its dynamic flow tracing approach with static analysis of the 
inspected applications. To my knowledge, such tool currently does not exist.
However, the Androguard\footnote{code.google.com/p/androguard} project provides
means of decompilation and static analysis of Android applications.

My previous experience includes a UROP internship at the Computer Lab
in the summer of 2011. Students worked on security-related projects on
the Android platform, an application for encrypted text-messaging in my
case. During my other internship in the summer of 2012, I worked on Java PathFinder,
a software-verification tool for JVM. My plugin verified correct usage of 
physical units in scientific computation by runtime assignment of attributes 
with unit information to numerical values in the memory, and their propagation 
during arithmetic operations, which can be regarded as a form of flow tracing. 

\section*{Substance and Structure of the Project}

The main outcome of this project will be a desktop application written 
in Java, capable of instrumenting given Dalvik executables, known as DEX
files, according to options specified via command-line arguments or via
graphical user interface.

%It will use the ASMDEX \cite{asm.ow2.org/asmdex-index.html} library to 
%parse, modify and reconstruct Dalvik bytecode. 

The development of a dependable code instrumentation, which will reliably 
identify sources and sinks, and which will propagate the taint throughout 
the application will form essential part of the project. This requires 
detailed analysis of the effects of each instruction supported by the 
Dalvik VM, and thorough testing. Special attention will need to be given 
to correct instrumentation of the exception mechanism.

Data sources and sinks will be specified as an XML file containing a list
of classes and/or methods inside the Android library, accessing which
should trigger the tainting mechanism. The same will apply to so-called 
sanitizing methods, calling which should not propagate the taint. These
are, for example, the hashing functions. Specifications of commonly used 
sources, sinks and sanitizers will be part of the distribution.

Typical overhead of dynamic data-flow analysis by instruction-level 
instrumentation varies between 3 and 35 times of the original speed. While 
there is very limited room for making the instrumentation itself more 
efficient, time can easily be invested into static analysis of the code which 
will identify the parts of the code that do not need to be instrumented, 
i.e. code outside all paths from sources to sinks. 

It will also be possible to extract executable code from applications 
installed on a connected device by executing tools from the Android SDK.
%and by calling apktool \cite{code.google.com/p/android-apktool}. 
After the instrumentation, the modified application will be repackaged and
uploaded back to the device. Since all packages need to be signed and there
is no access to the original key, packages will be signed by a key of user's
choice.

The development process will follow the philosophy of Test Driven 
Development. Thus a significant amount of time will be spent on building 
a robust set of unit tests to ensure high quality of code. JUnit will be 
used to test the internals of the instrumenting application, and a set of 
shell scripts will be created to automatically execute a collection
of tailored snippets of code on an actual Dalvik virtual machine, 
comparing their output with the output of their instrumented counterparts.

\subsubsection*{Implicit-flow analysis}

TaintDroid focuses strictly on tracing the explicit flow of information, 
by propagating the taint when data-handling CPU instructions are used. 
But information can leak via implicit flow (conditional branching 
instructions) as well. Consider the following example:

\begin{verbatim}
int sensitiveData = getSensitiveData(); // tainted variable
int leakedData = 0; // taintless variable

while (sensitiveData--)
    leakedData++;

output(leakedData); // leakedData still not tainted
\end{verbatim}

In this case, \verb|leakedData| has not been tainted because there is 
no explicit information flow from \verb|sensitiveData| into it. Instead, 
\verb|sensitiveData| effects the control flow of the program, leaking
the information into \verb|leakedData| implicitly. 

Leakage via implicit flow is best identified by static analysis of the 
source code of the program, a technique often used to analyse scripting
languages. Since the source code is not available for third-party 
applications on Android, it needs to be analysed dynamically on the 
instruction level. The downside of this solution is that the propagation 
rules must be rather conservative, leading to false positives. This is 
why TaintDroid developers decided not to include it into their
system-wide solution. Since this project will instrument applications
individually, the user will be given the choice of instrumentation 
including or excluding implicit-flow analysis.

\subsection*{Extensions}

Depending on the amount of time necessary to finish the core of the project,
extra effort will be put into extending the list of supported features. Each 
of the following extensions can be implemented separately and they are 
sorted according to the added benefit to the users in descending order.

\subsubsection*{Causality analysis}

While TaintDroid quite reliably identifies privacy policy violations,
it fails to provide information about their origin. The user might therefore 
have trouble distinguishing whether a violation happened in a 
background-running thread or as a result of interaction with the UI. 
By instrumenting the message loop of classes inheriting from the Android 
Activity class (equivalent of a window), this information could be provided.

\subsubsection*{Reflection}

One major limitation of analysis by bytecode instrumentation is that it 
cannot easily deal with reflection or dynamically loaded code. The core
project will simply warn the user when these are present in the processed
code, but a possible extension could further explore methods of handling 
these.

\subsubsection*{FastPath optimization}

The "A Low-Overhead Practical Information Flow Tracking System for 
Detecting Security Attacks" paper, which studied taint-based flow tracing 
in the context of x86 server applications, suggests a form of dynamic 
optimization called FastPath. It includes both the original and 
instrumented version of each method in the resulting executable, and
dynamically decides which will be called based on the presence of tainted 
arguments. Authors of the paper argued that only 2\% of typical 
function calls were tainted, which is why this optimisation had such  
impact on performance. Whether this is true in the context of Android mobile
applications as well is not clear. Before implementing FastPath, a short 
study of the number of tainted method calls should be conducted.

\section*{Success Criteria}

For the project to be deemed a success the following items must be
successfully completed.

\begin{enumerate}

\item Application must communicate with Android devices and be able to
      download/upload application packages (APKs).

\item Application must be able to extract content of APK files
      and to repackage them after instrumentation.

\item Specification of sources, sinks and sanitizers has to be designed
      and these properly identified by the instrumenting application.

\item Instrumentation of DEX files for both explicit and implicit flow 
      analysis needs to be designed, implemented and shown to propagate 
      taint correctly.

%\item Static analysis of execution paths between sources and sinks 
%      must be used to optimise performance.

\item Capabilities of the information-flow analysis must be demonstrated
      on examples of real applications.
      
\item Advantages and limitations of system-level integration versus 
      per-application bytecode instrumentation for the purpose of 
      taint-based flow analysis must be appraised.

\item Performance must be compared to TaintDroid and overhead measured
      against the original code.

\end{enumerate}

\section*{Plan of work}

\begin{itemize}

    \item{\textbf{Fortnight 1 (\nicefrac{2}{2} October 2012)}: \\
          Processing of a DEX file, analysis of its content, saving the 
          content to another file.}
    \item{\textbf{Fortnight 2 (\nicefrac{1}{2} November 2012)}: \\
          Instrumentation of trivial instructions for explicit-flow analysis,
          identification of sources, sinks and sanitizers.}
    \item{\textbf{Fortnight 3 (\nicefrac{2}{2} November 2012)}: \\
          Communication with Android devices, APK repackaging.}
    \item{\textbf{Fortnight 4 (\nicefrac{1}{2} December 2012)}: \\
          Instrumentation of more complex instructions for explicit-flow
          analysis, testing on real applications.}
    \item{\textbf{Fortnight 5 (\nicefrac{2}{2} December 2012)}: \\
          Progress report, time for course revision.}
    \item{\textbf{Fortnight 6 (\nicefrac{1}{2} January 2013)}: \\
          Implicit-flow analysis instrumentation, assessment of its impact.}
    \item{\textbf{Fortnight 7 (\nicefrac{2}{2} January 2013)}: \\
          Performance optimisation, benchmarking.}
    \item{\textbf{Fortnight 8 (\nicefrac{1}{2} February 2013)}: \\
          Automated testing on a large sample of applications.}
    \item{\textbf{Fortnights 9 \& 10 (\nicefrac{2}{2} February 2013, 
                                       \nicefrac{1}{2} March 2013)}: \\
          Extra time in case of development delay, work on extensions
          otherwise.}
    \item{\textbf{Fortnights 11 \& 12 (\nicefrac{2}{2} March 2013, 
                                       \nicefrac{1}{2} April 2013)}: \\
          Dissertation writing up, first draft given to the supervisor and
          the Director of Studies by the beginning of Easter term.}
    \item{\textbf{Fortnights 13 \& 14 (\nicefrac{2}{2} April 2013, 
                                       \nicefrac{1}{2} May 2013)}: \\
         Incorporating feedback into the dissertation.}
    
%    \item{\textbf{Fortnight 1:} Instrumenting application set-up; 
%          application-extraction and repackaging scripts; DEX processing, 
%          copying DEX content into a new file}
%   \item{\textbf{Fortnight 2:} Instrumentation of trivial instructions;
%         specification and identification of sources and sinks}
%   \item{\textbf{Fortnight 3:} Static analysis and instrumentation of 
%         exceptions; time for testing}
%   \item{\textbf{Fortnight 4:} Implicit-flow instrumentation; assessment 
%         of its impact}
%   \item{\textbf{Fortnight 5:} Performance optimisations}
%   \item{\textbf{Fortnight 6:} Causality analysis}
%   \item{\textbf{Fortnight 7:} Graphical user interface}
%   \item{\textbf{Fortnight 8:} Large-sample testing and evaluation}
%   \item{\textbf{Rest:} Dissertation}
\end{itemize}

\section*{Resources Declaration}

Development will require computers with the Android SDK installed. 
Testing on a large sample of applications will be executed on a 
PWF machine, but most of the work will be done on my personal computer 
(Asus UL20A laptop, Intel Core 2 Duo 1.3GHz CPU, 4GB RAM, 320GB HDD). 
I accept full responsibility for this machine and I have made contingency 
plans to protect myself against hardware and/or software failure.

Source code will be managed by the Git revision control system, with
the repository being hosted on GitHub. Storing the source code in the
cloud, together with the distributed nature of Git should provide
good protection against data loss.

Large collection of infected applications have been obtained from the
Android Malware Genome Project\footnote{www.malgenomeproject.org}. 
Sufficient disc allocation of 3GB will be needed on the PWF to store
a snapshot of the repository. 

An Android smartphone will be supplied by the supervisor\footnote
{Dr Alastair Beresford, arb33@cam.ac.uk} to test the project on. In case 
of problems, the Android emulator can be used instead.

%\printbibliography
\end{document}
